\documentclass[11pt,letterpaper]{article}

\usepackage[margin=1in]{geometry}
\usepackage{termcal}
\usepackage{enumitem}
\usepackage[colorlinks=true, allcolors=blue]{hyperref}
\usepackage{color}
\usepackage{multirow}
\usepackage{multicol}

\newcommand{\todo}[1]{\textcolor{red}{TODO: #1}}

\title{16.685: Spacecraft Attitude Determination and Control}
\author{Spring 2026}
\date{}

\begin{document}

\maketitle

%\noindent 
%\textbf{Course website:} \href{https://optimalcontrol.ri.cmu.edu/}{optimalcontrol.ri.cmu.edu}

\section*{Course Description}

This course is all about attitude: pointing, slewing, spinning, and (occasionally, but hopefully not too often...) tumbling. We will discuss:
\begin{itemize}
\item Attitude representations and how to effectively parameterize attitude in different situations.
\item How to model and simulate spacecraft attitude dynamics.
\item How to estimate attitude from sensor measurements.
\item How to manage a spacecraft's attitude both with passive physics and closed-loop control.
\end{itemize}
The course will be based around a project in which each student analyzes the attitude determination and control system (ADCS) for a spacecraft mission.

\medskip
\noindent
\textbf{Prerequisites:} Strong linear algebra skills, experience with a high-level programming language like Python, MATLAB, or Julia, and basic familiarity with ordinary differential equations.

\section*{Instructor}

\begin{center}
\begin{tabular}{l l l}
	Prof. Zac Manchester & \textbf{Email:} \href{mailto:zacm@mit.edu}{zacm@mit.edu} & \textbf{Office:} 16-243
\end{tabular}
=\end{center}

\section*{Logistics}

\begin{itemize}
	\item Lectures will be held \todo{time} Eastern time in \todo{place}. Lectures will also be live streamed on zoom and recorded for later viewing.
	%\item Recitation will be held Fridays on Zoom at 11 AM.
	\item Office hours will be \todo{based on survey}.
	\item Homework assignments will be due by \todo{due date}. Two weeks will be given to complete each assignment.
	\item Short quizes will occasionally be assigned to assess student comprehension. These will be graded only based on completion.+
	% \item GitHub will be used to distribute assignments and GradeScope will be used for submissions.
	\item Slack will be used for general discussion and Q\&A outside of class and office hours.
	\item There will be no exams. Instead, students will compile a final report based on analysis done in each homework assignment.
\end{itemize}

\newpage 
\section*{Learning Objectives}
By the end of this course, students should be able to do the following:
\begin{enumerate}
	\item Model and simulate the attitude dynamics of a spacecraft
	\item Analyze stability of spacecraft attitude dynamics
	\item Estimate the attitude of a spacecraft from sensor measurements
	\item Design feedback controllers to stabilize spacecraft attitude
	\item Design slewing maneuvers to point a spacecraft at a desired target
	\item Characterize the performance of a closed-loop attitude control system
	\item Select sensors and actuators to meet mission requirements
\end{enumerate}

\section*{Learning Resources}

There is no textbook for this course. Video recordings of lectures and lecture notes will be posted online. The following books are good references for some of the topics discussed in the course that you may want to refer to, but are not required:

\begin{enumerate}
	\item P. Hughes, \textit{Spacecraft Attitude Dynamics}, Dover, 2004.
	\item F. Markley and J. Crassidis, \textit{Fundamentals of Spacecraft Attitude Determination and Control}, Springer, 2014.
	\item J. Wertz, \textit{Spacecraft Attitude Determination and Control}, Kluwer, 1978.
\end{enumerate}

\section*{Assignments}

Every two weeks students will be asked to complete a homework assignment that includes analysis of some aspect of the ADCS design for their chosen spacecraft mission. A write-up of this analysis will be reviewed by the instructor and feedback will be returned to students the following week. At the end of the semester, these assignments will be compiled into a final report. There will be no exams in this course.

\section*{Grading}

Grading will be based on:
\begin{itemize}
	\item 50\% Final Report
	\item 40\% Homework
	\item 5\% Quizes
	\item 5\% Participation
\end{itemize}
Attendance during lectures is not required to earn a full participation grade. Students can also participate through any combination of office hours, Slack discussions, project presentations, and by offering constructive feedback about the course to the instructors.


\section*{Course Policies}

\textbf{Late Homework:} Students are allowed a budget of 5 late days for turning in homework with no penalty throughout the semester. They may be used together on one assignment, or separately on multiple assignments. Beyond these six days, no other late homework will be accepted.

\section*{Tentative Schedule}

\begin{tabular}{c|c|c|c}
	Week & Dates & Topics & Assignments \\
	\hline
	\multirow{2}{*}{1} & Feb 3 & 
        Course Overview, \& Attitude Intro & Survey \\
	 & Feb 5 & Representations, SO(3), and Quaternions &  HW1 Out\\
	\hline
	\multirow{2}{*}{2} & Feb 10 &
        Rigid Body Dynamics &  \\
	 & Feb 12 & Gyrostat Dynamics &  \\
	\hline
	\multirow{2}{*}{3} & Feb 17 &
        Damping and Environmental Perturbations & HW1 Due \\
	 & Feb 19 &  & HW2 Out \\
	\hline
	\multirow{2}{*}{4}  & Feb 24 &
        Spinning Spacecraft and Stability &  \\
	 & Feb 26 & Numerical Simulation &  \\
	\hline
	\multirow{2}{*}{5}  & Mar 3 &  Attitude Determination Sensors & HW2 Due \\
	 & Mar 5 & TRIAD & HW3 Out \\
	\hline
	\multirow{2}{*}{6}  & Mar 10 & Statistical Estimation &  \\
	 & Mar 12 & Optimizing with Attitude &  \\
	\hline
	\multirow{2}{*}{7}  & Mar 16 & Wahba's Problem
         & HW3 Due \\
	 & Mar 18 & Nonlinear Least-Squares \& Convex Relaxations & \\
	\hline
	\multirow{2}{*}{8}  & Mar 24 & 
        \textcolor{red}{No Class} & \\
	 & Mar 26 & \textcolor{red}{No Class} &   \\
	\hline
	\multirow{2}{*}{9}  & Mar 31 & Kalman Filters
         & HW4 Out \\
	 & Apr 2 & Multiplicative \& Invariant EKF &  \\
	\hline
	\multirow{2}{*}{10}  & Apr 7 & Passive Attitude Control Methods
         &  \\
	 & Apr 9 & Attitude Control Actuators &   \\
	 \hline
	\multirow{2}{*}{11}  & Apr 14 & Feedback Controllers
         & HW4 Due \\
	 & Apr 16 & Stabilization and Tracking &  HW5 Out \\
	 \hline
	\multirow{2}{*}{12}  & Apr 21 & Designing Slew Maneuvers 
         &   \\
	 & Apr 23 & Optimal Control &   \\
	 \hline
	\multirow{2}{*}{13}  & Apr 28 & Performance Analysis
         & HW5 Due \\
	 & Apr 30 & Calibration &   \\
	 \hline
	\multirow{2}{*}{14}  & May 5 & Advanced Topics
         &  \\
	 & May 7 & Case Studies &   \\
	 \hline
	\multirow{2}{*}{14}  & May 12 &
        Case Studies &  \\
	 &  &  &   \\
\end{tabular}


\end{document}
